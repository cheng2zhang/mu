%\documentclass{revtex4-1}
% options: aip, jcp, reprint, preprint
%\documentclass[preprint]{revtex4-1}
%\documentclass[notitlepage, preprint]{revtex4-1}
\documentclass[aip,jcp,reprint,superscriptaddress]{revtex4-1}

\usepackage{amsmath}
\usepackage{amsthm}
\usepackage{mathrsfs}
\usepackage{graphicx}
\usepackage{dcolumn}
\usepackage{bm}
\usepackage{multirow}
\usepackage{hyperref}
\usepackage{tikz}
%\usepackage{setspace}

\tikzstyle{blackdot}=[circle,draw=black,fill=black,
                      inner sep=0pt,minimum size=1.5mm]
\tikzstyle{whitedot}=[circle,draw=black,fill=white,
                      inner sep=0pt,minimum size=1.5mm]
\tikzstyle{combine}=[green!30!black, very thick, densely dotted]
\tikzstyle{concat}=[blue!40!black, very thick, densely dashed]

%\renewcommand{\theequation}{N\arabic{equation}}

\newtheorem{defn}{Definition}
\newtheorem{thrm}{Theorem}
\newtheorem{lemm}[thrm]{Lemma}
\newtheorem{prop}[thrm]{Proposition}

\newcommand{\vct}[1]{\mathbf{#1}}
\providecommand{\vr}{} % clear \vr
\renewcommand{\vr}{\vct{r}}
\newcommand{\vk}{\vct{k}}
\newcommand{\vR}{\vct{R}}
\newcommand{\dvr}{d\vr}
\newcommand{\dvk}{\frac{d\vk}{(2\pi)^D}}
\newcommand{\tdvk}{\tfrac{d\vk}{(2\pi)^D}}

% add a superscript ``ex''
\newcommand{\supex}[1]{ { { #1 }^{ \mathrm{ex} } } }
\newcommand{\Pex}{\supex{P}}
\newcommand{\Fex}{\supex{F}}
\newcommand{\muex}{\supex{\mu}}
\newcommand{\kex}{\supex{\kappa}}

\newcommand{\muv}{\mu^{(v)}}
\newcommand{\muc}{\mu^{(c)}}
\newcommand{\mux}{\mu^{(\xi)}}
\newcommand{\muexv}{\mu^{\mathrm{ex}, (v)}}
\newcommand{\muexc}{\mu^{\mathrm{ex}, (c)}}
\newcommand{\muexx}{\mu^{\mathrm{ex}, (\xi)}}
\newcommand{\mutex}{\mu^{ \mathrm{ex} }_t}
\newcommand{\secref}[1]{Sec. \ref{#1}}

\newcommand{\llbra}{[\![}
\newcommand{\llket}{]\!]}





\begin{document}

\title{Approximate expressions of chemical potential from integral equations}

\begin{abstract}
  Several expressions of computing
  chemical potentials
  from integral equations
  are reviewed.
  %
  A simple strategy is proposed to approximately
  compute chemical potential
  from these routes.
\end{abstract}
\maketitle





\section{Introduction}



For a given liquid,
the chemical potential, $\mu$,
gives the free energy of adding a particle into the system.
%
The chemical potential can be computed from
molecular dynamics (MD) and
Monte Carlo (MC) simulations,
by e.g., Widom's insertion formula
or thermodynamic integration.
%
The simulations, however, can be expensive
and yield little insight.
%
Here, we consider a cheaper alternative
by using integral equations
from liquid state theory\cite{hansen}.



The method of integral equations
focuses on computing approximate correlation functions,
such as the pair correlation function, $g(r)$,
and the direct correlation function, $c(r)$.
The chemical potential is then expressed
as a functional of the correlation functions.
%
Although different expressions exist,
they should ideally yield the same value.
%
In practice, however,
approximate integral equations
are internal inconsistent,
and different expressions (or routes)
yield different values.
%
% integral equation is approximate.
% this leads to internal inconsistency.
% the inconsistency leads to different values from different routes
%



Here, we review
a few expressions of computing the chemical potential.
%
The two well-known expressions
for computing the compressibility and (virial) pressure
can be adapted here for computing the chemical potential.
%
Additionally,
the chemical-potential can be obtained
from an integral over a charging parameter, $\xi$,
which controls the interaction
of a distinct particle with other particles.
%
In the last case,
the resulting value depends on the definition of $\xi$,
since the integral generally does not yield a state function.
%
Further,
for each chemical-potential route,
there is a corresponding generalized energy route.
%
The above expressions all involve integrals
over some parameter,
e.g., the density, $\rho$,
or the charging parameter, $\xi$.
%
Here, we also attempt to give approximate expressions
that avoid such integrals.




\section{Methods}





\subsection{Basics of integral equations}



We first review the basics of integral equations
in liquid state theory.
%
We shall use the standard notations:
%
$\phi(r)$
is the pair potential
for two particles separated by $r$;
%
$f(r) = e^{ -\beta \, \phi(r) } - 1$
is the Mayer $f$-function
with
$\beta = 1/(k_B \, T)$
being the inverse temperature;
%
$c(r)$, $t(r)$, and $h(r) = g(r) - 1 = c(r) + t(r)$
are the direct, indirect,
and total correlation functions, respectively;
%
$y(r) = g(r) \, e^{\beta \phi(r)}$
is the cavity distribution function;
%
and
$B(r) = \log y(r) - t(r)$
is the bridge function.



If the bridge function, $B(r)$, is given
approximately as a functional of $t(r)$,
then there are only two independent correlation functions,
e.g., $c(r)$ and $t(r)$, which can be determined
by the following two equations:
%
the Ornstein-Zernike (OZ) relation
\begin{equation}
  t(r) = \rho \, (c * h)(r)
       = \rho \, [c * (t + c)] (r),
  \label{eq:oz}
\end{equation}
%
with $*$ denoting a convolution
$(a * b)(\vr) = \int d\vr' \, a(\vr') \, b(\vr - \vr')$,
%
and the closure
%
\begin{align}
  c(r)
  &= e^{-\beta \phi(r)} \, y(r) - 1 - t(r) \notag \\
  &= [f(r) + 1] \, e^{t(r) + B(r)} - 1 - t(r).
  \label{eq:closure}
\end{align}
%
where $f(r) \equiv e^{-\beta \phi(r)} - 1$.



Two commonly used closures are
the Percus-Yevick (PY)
and hypernetted chain (HNC).
%
In the former case,
%
\begin{align}
y(r) = 1 + t(r),
\label{eq:PY}
\end{align}
%
or, equivalently,
$B(r) = \log[1 + t(r)] - t(r)$.
%
In the latter case,
%
\begin{align}
y(r) = \exp t(r),
\label{eq:HNC}
\end{align}
%
or, equivalently,
$B(r) = 0$.



Many interpolations of the above two closures
improve the accuracy.
%
Among them, we only consider
the Hutchinson-Conkie (HC) closure\cite{hutchinson1971}:
%
\begin{equation}
y(r) = \left[1 + s \, t(r)\right]^{1/s},
\label{eq:HC}
\end{equation}
with $s = 0.84$.





\subsection{
\label{sec:cvroutes}
Compressibility and virial routes}




We first show
how to compute chemical potential
from the compressibility and virial routes.
%
Chemical potential is related to pressure by
%
\begin{align}
\partial_\rho \mu^\mathrm{ex}
=
(\partial_\rho P^\mathrm{ex}) / \rho,
\label{eq:mu_P}
\end{align}
%
where the superscript ``ex''
means the excess (non-ideal-gas) part.
%
Thus,
we have
%
\begin{align}
\mu^\mathrm{ex}
=
\int_0^\rho
  \frac{ d\rho' }{ \rho' }
  \frac{ \partial P(\rho') }{ \partial \rho' }.
\label{eq:mu_intrho}
\end{align}
%



Particularly,
for the compressibility route,
we have
%
\begin{align}
\beta \, \partial_\rho P_c^\mathrm{ex}
=
-\rho \, \int d\vr \, c(r),
\label{eq:dP_c}
\end{align}
%
where,
$\beta = 1/(k_B \, T)$
is the inverse temperature
(with $k_B$ being the Boltzmann constant).
%
Using this in Eq. \eqref{eq:mu_P}
yields
%
\begin{equation}
\beta \, \partial_\rho \mu_c^\mathrm{ex}
=
-\int d\vr \, c(r)
\label{eq:dmu_c}
\end{equation}
%
By integrating over $\rho$,
we get the compressibility-route chemical potential:
%
\begin{equation}
\beta \, \mu_c^\mathrm{ex}
=
-\int_0^\rho d \rho' \int d\vr \, c(r).
\label{eq:mu_c}
\end{equation}
%
Note that $c(r)$ implicitly depends on the density, $\rho'$.



Similarly, for the virial route,
we have
%
\begin{align}
\beta \, P_v^\mathrm{ex}
&=
-\frac{ \rho^2 } { 2 \, D }
\, \int d\vr \, r \phi'(r) \, g(r)
\notag \\
&=
\frac{ \rho^2 } { 2 \, D }
\, \int d\vr \, r f'(r) \, y(r),
\label{eq:P_v}
\end{align}
%
where $D$ is the dimension.
%
By using Eq. \eqref{eq:mu_P},
we get
%
\begin{align}
\beta \, \partial_\rho \mu_v^\mathrm{ex}
&=
\frac{ 1 } { D }
\, \int d\vr \, r f'(r)
\left[
  y(r)
  +
  \frac{\rho}{2} \, \partial_\rho y(r)
\right].
\label{eq:dmu_v}
\end{align}
%
Thus, the virial-route chemical potential reads
%
\begin{equation}
\beta \, \mu_v^\mathrm{ex}
=
\frac{1}{D}
\int_0^\rho d\rho'
  \int d\vr \,
    r f'(r) \,
    \left[
    y(r) + \frac{\rho'}{2} \partial_{\rho'} y(r)
    \right].
\label{eq:mu_v}
\end{equation}





\subsubsection{Approximate functional}





The $\rho$ integrals
in Eqs. \eqref{eq:mu_c} and \eqref{eq:mu_v}
are cumbersome.
%
Below we make further approximations
to avoid them.



We first assume that the chemical potential satisfies
%
\begin{align}
  \beta \muex
  =
  -\rho \, \int d\vr \,
    \left[
        c(r) - B(r)
      - \frac{1}{2} \, h(r) \, t(r)
      - \gamma \, \Delta(r)
    \right],
  \label{eq:muex_trial}
\end{align}
%
where
$\Delta(r)$ is a trial function,
and
$\gamma$ is a parameter to be determined.
%
This functional form is inspired by
the exact virial- or chemical-potential-route formula
for the HNC equation\cite{
morita1960, morita1960I, singer1985}:
%
\begin{equation}
  \beta \muex
  =
  -\rho \, \int d\vr \,
    \left[
      c(r) - \frac{1}{2} \, h(r) \, t(r)
    \right].
  \label{eq:muex_hnc}
\end{equation}





\subsubsection{Parameter determination}





To find a proper $\gamma$,
we insist that the trial functional form
given by Eq. \eqref{eq:muex_trial}
also satisfies
Eq. \eqref{eq:dmu_c} or \eqref{eq:dmu_v},
%
This condition demands
%
\begin{align}
\gamma
=
\frac{
  Y(r) + (1 + \rho \, \partial_\rho) \, X(r)
}
{
  (1 + \rho \, \partial_\rho) \, \Delta(r)
},
\label{eq:gamma_murho}
\end{align}
%
where
%
\begin{align}
X(r) = c(r) - B(r) - \frac{1}{2} h(r) \, t(r),
\label{eq:X}
\end{align}
%
and
\begin{align}
Y_c(r) = -c(r)
\label{eq:Yc}
\end{align}
for the compressibility route, or
\begin{align}
Y_v(r) = \frac{ 1 }{ D }
\int d\vr \, r f'(r) \,
\left[
	y(r) + \frac{\rho}{2} \, \partial_\rho y(r)
\right]
\label{eq:Yv}
\end{align}
for the virial route.





\subsubsection{Correction function}




We consider two possibilities for
the correction function, $\Delta(r)$.
%
The first candidate is
%
\begin{equation}
\Delta(r) = h(r) \, B(r).
\label{eq:Delta_hB}
\end{equation}
%
In the low density limit,
one can show by graphical analysis that
$\gamma$ from Eq. \eqref{eq:gamma_murho}
is $2/3$.



The second candidate is
%
\begin{equation}
\Delta(r) = -B(r).
\label{eq:Delta_B}
\end{equation}



However,
for the HNC equation,
$B(r) \equiv 0$,
and in order to compute
the compressibility-route value,
we use an effective bridge function
%
\begin{equation}
B^*(r) = \log\left[ 1 + t(r) \right] - t(r),
\label{eq:Beff_hnc}
\end{equation}
%
and use $B^*(r)$ in place of $B(r)$
in Eqs.
\eqref{eq:muex_trial},
\eqref{eq:X},
\eqref{eq:Delta_hB},
and
\eqref{eq:Delta_B}.
%
%\begin{equation}
%\Delta^\mathrm{HNC}(r)
%=
%c(r) - \left[1 + \rho \, (f * f)(r)\right] \, f(r).
%\label{eq:Delta_hnc}
%\end{equation}






\subsection{
\label{sec:xiroute}
Chemical-potential route}





We now turn to the chemical-potential route.
%
First, we need to
select a designated particle 1 from the fluid
and define a charging parameter, $\xi$,
to control its interaction with other particles.
%
When $\xi = 0$,
particle 1 does not interact with other particles.
%
When $\xi = 1$,
particle 1 interacts normally with other particles.



Then,
by thermodynamic integration,
we have
%
\begin{align}
\partial_\xi ( \beta \mu^\mathrm{ex} )
&=
\langle \partial_\xi (\beta \Delta U) \rangle_\xi
\notag \\
&=
\rho \, \int d\vr \,
\beta \, \partial_\xi \phi(r; \xi) \, g(r; \xi),
\label{eq:dmuti0}
\end{align}
%
where,
$\Delta U = \sum_{i = 2}^N \phi(r_{1i}; \xi)$
is the interaction energy of particle 1
and other particles.
%
Integrating over $\xi$ yields
the chemical-potential-route formula:
%
\begin{align}
\beta \, \mu^\mathrm{ex}
&=
\rho \int_0^1 d\xi \,
\int d\vr \, \beta \, \partial_\xi \phi(r; \xi) \,
g(r; \xi)
\notag \\
&=
-\rho \int_0^1 d\xi \,
\int d\vr \, \partial_\xi f(r; \xi) \,
y(r; \xi).
\label{eq:muti0}
\end{align}


%\subsubsection{Partial integration}

The $\xi$ integral can be partially avoided.
%
After partial integration and some algebra,
Eq. \eqref{eq:dmuti0} becomes
%
\begin{align}
\partial_\xi ( \beta \mu^\mathrm{ex} )
&=
-\rho \, \partial_\xi \int d\vr \,
    \left[
      c(r; \xi)
    - B(r; \xi)
    - \tfrac{1}{2} h(r; \xi) \, t(r; \xi)
    \right]
\notag \\
&\phantom{=}
  +\rho \, \int d\vr \,
  h(r; \xi) \,
  \partial_\xi B(t; \xi),
\label{eq:dmuti}
\end{align}
%
where we have used
\begin{align*}
\int d\vr \,
h(r; \xi) \, \partial_\xi t(r; \xi)
=
\partial_\xi \int d\vr \,
\frac{1}{2} \, h(r; \xi) \, t(r; \xi).
\end{align*}
%
Then, integrating over $\xi$ yields
%
\begin{align}
\beta \mu^\mathrm{ex}
&=
-\rho \int d\vr
\left[
  c(r) - B(r) - \frac{1}{2} t(r) \, h(r)
\right]
\notag \\
&
+ \rho \int_0^1 d\xi \, \int d\vr
  \, h(r; \xi) \, \partial_\xi B(r; \xi),
\label{eq:muti}
\end{align}
%
where
$a(r)$ is short for $a(r; \xi = 1)$
for function $a$.



The trial functional form
given by Eq. \eqref{eq:muex_trial}
is based on Eq. \eqref{eq:muti}.
%
The comparison shows that
that the correction $\Delta(r)$ is $O(\rho^2)$,
which is small at a low density.




\subsubsection{Approximate formula}





Similar to the previous two routes,
an approximate formula for chemical potential
can be developed.
%
Below we assume that
the functional form of Eq. \eqref{eq:muex_trial}
holds for any $\xi$.
%
Then,
in order to satisfy Eq. \eqref{eq:dmuti},
we have
%
\begin{align}
\gamma
=
\frac
{
  \int d\vr \,
  h(r; 1) \, \partial_\xi B(r, 1)
}
{
  \int d\vr \,
  \partial_\xi \Delta(r; 1)
}.
\label{eq:gamma_xi_Delta}
\end{align}
%
Particularly,
for the $\Delta$ given by Eq. \eqref{eq:Delta_hB},
we have
%
\begin{align}
\gamma
=
\frac
{
  \int d\vr \, h(r; 1) \, \partial_\xi B(r, 1)
}
{
  \int d\vr \,
  \left[
  B(r; 1) \, \partial_\xi h(r; 1)
  +
  h(r; 1) \, \partial_\xi B(r; 1)
  \right]
}.
\label{eq:gamma_xi}
\end{align}
%





\subsubsection{Charging parameter}




The result of Eq. \eqref{eq:muti0}
depends critically on
the definition
of the charging parameter, $\xi$.
%
%The choice also affects the accuracy
%of the approximate formula
%given by Eqs. \eqref{eq:muex_trial} and
%\eqref{eq:gamma_xi_Delta}.
%
%
%
The most natural choice is perhaps
given by
%
\begin{align}
\phi(r; \xi)
=
\xi \, \phi(r).
\label{eq:xi_phi}
\end{align}
%
It, however, does not apply
to the hard-sphere fluid.
%
Several alternatives are listed below.
%
First,
we may let
%
\begin{align}
f(r; \xi)
=
f(r/\xi),
\label{eq:xi_r}
\end{align}
%
which makes $\xi$ the scaling factor
of the interaction radius.
%
It was shown that
the choice was rather accurate
for the hard-sphere fluid\cite{
santos2012, santos2013}.
%
This route is referred to as
the $\xi r$ route below.



Second,
we may let
%
\begin{align}
f(r; \xi)
=
\xi \, f(r).
\label{eq:xi_f}
\end{align}
This choice,
referred to as the $\xi f$ route,
is computationally convenient,
for $\partial_\xi f(r; \xi) = f(r)$.



Finally,
the choice of
%
\begin{align}
t(r; \xi)
=
\xi \, t(r),
\label{eq:xi_t}
\end{align}
%
which is referred to as the $\xi t$ route,
is also occasionally used,
for it often allows
a closed-form expression of
chemical potential\cite{
attard1991}.
%
Note that,
because of the linearity
of the OZ relation:
$t(r; \xi) =
\rho \int d\vr' c(r'; \xi) *
h(|\vr - \vr'|)$,
this choice is equivalent to
the $\xi c$ and $\xi h$ routes:
\[
c(r; \xi)
=
\xi \, c(r),
\]
and
\[
h(r; \xi)
=
\xi \, h(r).
\]





\subsection{
\label{sec:eroute}
Generalized energy route}



\newcommand{\eps}{\varepsilon^\mathrm{ex}}
\newcommand{\Eps}{\mathcal E}
\newcommand{\U}{\mathcal U}



The charging parameter, $\xi$,
from the chemical-potential route
applies to only
the interaction with a single particle.
%
We can similarly defined a parameter, $\lambda$,
that applies to the interaction of
all particle pairs.
%
This leads to the generalized energy route.



For a system with a fixed volume, we have
%
\begin{align*}
d \log Z
=
- \Eps \, d\lambda
- \beta \, \mu \, dN,
\end{align*}
%
where
\begin{align*}
Z
\equiv
\int d\vr_1 \cdots d\vr_N \,
\exp[ -\U_\lambda(\vr_1, \dots, \vr_N) ],
\end{align*}
%
and
%
\begin{align*}
\Eps
&\equiv
\int d\vr_1 \cdots d\vr_N \,
\frac{ \partial \U_\lambda }
     { \partial \lambda }
\frac{ \exp[ -\U_\lambda(\vr_1, \dots, \vr_N) ] }
     { Z }.
\end{align*}
%
Thus,
\[
\partial_\lambda ( \beta \, \mu )
=
-\partial_\lambda
\partial_N
\log Z
=
\partial_N \Eps,
\]
or
%
\begin{equation}
\partial_\lambda (\beta \, \mu^\mathrm{ex})
=
\partial_\rho (\rho \, \eps),
\label{eq:mu_e}
\end{equation}
%
where
$\eps = \Eps^\mathrm{ex}/N$
is the internal energy per particle.
%
Thus,
%
\begin{align}
\beta \, \mu^\mathrm{ex}
&=
\int_0^1 \,
  d\lambda \,
  \partial_\rho (\rho \, \eps)
\notag \\
&=
-\int_0^1
  d\lambda \, \rho
  \int d\vr \,
    \left[
      \partial_{\lambda} f(r)
    \right]
    \left(
    1 + \tfrac{ \rho }{ 2 } \, \partial_\rho
    \right)
    y(r),
\label{eq:muex_intlam}
\end{align}
%
where,
the inner integral is evaluated at $\lambda$,
and
we have used the relation\cite{hansen},
%
\begin{equation}
\rho \, \eps
=
\frac{ \rho^2 } { 2 }
\int d\vr \, \partial_\lambda [\beta \, \phi(r)] \, g(r)
\label{eq:e_int}
\end{equation}
%
and
$\partial_\lambda [\beta \, \phi(r)] \, g(r)
 = -[ \partial_\lambda f(r) ] \, y(r)$.




\subsubsection{Parameter $\lambda$}



The parameter, $\lambda$,
is similar to the charging parameter, $\xi$.
%
We discuss a few choices below.
First, for a linear scaling,
$\U = \lambda \, \beta \, U$,
we get the energy route.
%
This choice is inapplicable
to the hard-sphere fluid,
and is not discussed further.



Second,
consider the scaling of the interaction radius:
$\phi_\lambda(r) = \phi(r/\lambda)$,
which is referred to
as the $\lambda r$ route.
%
One can show that
Eq. \eqref{eq:muex_intlam}
yields the same result as
the virial-route formula,
given by
Eqs.
\eqref{eq:muex_trial},
\eqref{eq:gamma_murho},
\eqref{eq:X},
and
\eqref{eq:Yv}.



Generally,
every choice of $\xi$
can be borrowed to define $\lambda$.
%
Thus,
we may consider
$f(r; \lambda) = \lambda \, f(r)$,
$c(r; \lambda) = \lambda \, c(r)$,
$h(r; \lambda) = \lambda \, h(r)$,
and
$t(r; \lambda) = \lambda \, t(r)$,
%
which will be referred to as the
$\lambda f$,
$\lambda c$,
$\lambda h$,
and
$\lambda t$
routes, respectively.
%
However, unlike the chemical-potential-route cases,
the last three options will yield different results.





\section{Results}




The integral equations and
those the differential correlation functions
were solved using the MDIIS method\cite{kovalenko1999}
with three to five bases.



\subsection{Hard-sphere fluid}


We tested the formulae of chemical potential
on the three-dimensional hard-sphere fluid.



\subsubsection{PY}



We first discuss the results
from the PY equation.
%
For the compressibility and virial routes,
the approximate formula given by
Eq. \eqref{eq:muex_trial}
[with the parameters determined from
Eqs.
\eqref{eq:gamma_murho},
\eqref{eq:X},
\eqref{eq:Yc},
and
\eqref{eq:Yv}]
were good for low densities,
but slightly off for high densities,
as shown in Fig. \ref{fig:muhspycv}.
%
The correction function, $\Delta = -B$,
appeared to yield slightly better results than
$\Delta = h \, B$.



\begin{figure}[h]
  \makebox[\linewidth][c]{
    \includegraphics[angle=0, width=0.9\linewidth]{fig/muhspycv.pdf}
  }
  \caption{
    \label{fig:muhspycv}
    Compressibility and virial routes
    for the PY equation
    in the three-dimensional hard-sphere fluid.
    %
    Unless indicated otherwise,
    the values were computed
    from the approximate formula,
    Eq. \eqref{eq:muex_trial}.
    %
    VS12: twelve-term virial series
    with the virial coefficients obtained
    from simulations of Mayer sampling\cite{
    schultz2014}.
    %
    (a) .
    The exact values were taken from
    the analytic solution\cite{
    wertheim1963, wertheim1964, thiele1963,
    baxter1968, hansen}.
  }
\end{figure}



The three chemical-potential routes
all underestimated the chemical potential,
as shown in Fig. \ref{fig:muhspyxi}.
%
Among them,
the $\xi r$-route was the most accurate,
whereas the $\xi f$-route was the least.
%
The approximate formula,
Eq. \eqref{eq:muex_trial},
worked well for the correction
$\Delta = h B$,
but not for $\Delta = -B$.



\begin{figure}[h]
  \makebox[\linewidth][c]{
    \includegraphics[angle=0, width=0.9\linewidth]{fig/muhspyxi.pdf}
  }
  \caption{
    \label{fig:muhspyxi}
    Chemical-potential routes,
    $\xi r$, $\xi f$ and $\xi t$ routes,
    for the PY equation
    in the three-dimensional hard-sphere fluid.
    %
    Unless indicated otherwise,
    the values were computed
    from the approximate formula,
    Eq. \eqref{eq:muex_trial}.
    %
    VS12: twelve-term virial series
    with the virial coefficients obtained
    from simulations of Mayer sampling\cite{
    schultz2014}.
    %
    For the $\xi r$-route values,
    the $\xi$ integral
    for the interval $(0, 1/2)$
    used the exact solution\cite{
    santos2012, santos2013}
    instead of the approximation
    from the integral equation
    (although the different is negligible).
    %
    The reference values were obtained
    from Eq. \eqref{eq:muti0}.
  }
\end{figure}





The results from
the generalized energy routes
are shown in Fig. \ref{fig:muhspylam}.
%
The $\lambda r$ route
yielded identical results to the virial route.
%
The $\lambda c$ route
yielded more accurate results than
the corresponding chemical-potential route
(the $\xi c$ route,
which is equivalent to the $\xi t$ route),
but the $\lambda f$ route
yielded less accurate results than
the $\xi f$ route.
%
The $\lambda t$ and $\lambda h$ routes
were similar to the $\lambda c$ route
at low densities,
but they became unstable at high densities.
%
The approximate formula,
Eq. \eqref{eq:muex_trial},
did not work well
for the $\lambda h$, and $\lambda t$
routes at high densities.
%
Thus, a different trial function
or a better correction function, $\Delta$,
might be needed for these cases.




\begin{figure}[h]
  \makebox[\linewidth][c]{
    \includegraphics[angle=0, width=0.8\linewidth]{fig/muhspylam.pdf}
  }
  \caption{
    \label{fig:muhspylam}
    Generalized energy routes
    for the PY equation
    in the three-dimensional hard-sphere fluid.
    (a)
    The $\lambda r$ and
    $\lambda f$ routes.
    (b)
    The $\lambda c$,
    $\lambda h$,
    and
    $\lambda t$ routes.
    %
    Some high density points were missing
    because of numerical difficulty
    and/or intrinsic instability.
    For the approximate formula,
    Eq. \eqref{eq:muex_trial},
    only $\Delta = h \, B$ was used,
    for $\Delta = -B$ often led to
    unstable solutions.
    %
    VS12: twelve-term virial series
    with the virial coefficients obtained
    from simulations of Mayer sampling\cite{
    schultz2014}.
    %
    The reference values were obtained
    from Eq. \eqref{eq:muti0}.
  }
\end{figure}






\subsubsection{HNC}





For the HNC equation,
all chemical-potential routes,
all generalized energy routes
and the virial route
should yield the same value
as that given by
Eq. \eqref{eq:muex_hnc}.
%
For the $\xi r$, $\lambda c$,
and virial routes,
this agreement can be seen
from Fig. \ref{fig:muhshnc},
%
The compressibility-route results
were well approximated by
Eq. \eqref{eq:muex_trial}
[with the parameters determined from
Eqs.
\eqref{eq:gamma_murho},
\eqref{eq:X},
\eqref{eq:Yc}].
%


\begin{figure}[h]
  \makebox[\linewidth][c]{
    \includegraphics[angle=0, width=0.9\linewidth]{fig/muhshnc.pdf}
  }
  \caption{
    \label{fig:muhshnc}
    Chemical potential from the HNC equation
    for the three-dimensional hard-sphere fluid.
    %
    VS12: twelve-term virial series
    with the virial coefficients obtained
    from simulations of Mayer sampling\cite{
    schultz2014}.
  }
\end{figure}




\subsubsection{HC}





The HC equation, with $s = 0.84$,
was a more accurate integral equatoin than
both the PY and HNC equations.
%
As shown in Fig. \ref{fig:muhshc},
all three routes improved over
the PY or HNC counterparts.
%
Among them,
the compressibility route
was the most accurate.
%
The $\xi r$-route
was again better than the virial route
(or the $\lambda r$-route).
%





\begin{figure}[h]
  \makebox[\linewidth][c]{
    \includegraphics[angle=0, width=0.9\linewidth]{fig/muhshc.pdf}
  }
  \caption{
    \label{fig:muhshc}
    Chemical potential from the Hutchinson-Conkie (HC) equation
    with $s = 0.84$
    for the three-dimensional hard-sphere fluid.
    %
    VS12: twelve-term virial series
    with the virial coefficients obtained
    from simulations of Mayer sampling\cite{
    schultz2014}.
  }
\end{figure}




\subsubsection{Ideal $\gamma$}



For the approximate formulae
based on Eq. \eqref{eq:muex_trial},
the difference between routes
lies in the $\gamma$ value,
which was obtained from
Eq. \eqref{eq:gamma_murho}
or
\eqref{eq:gamma_xi_Delta}.
%
Here, we computed the ideal $\gamma$ values
that yield the exact $\mu^\mathrm{ex}$,
and the results are shown in Fig. \ref{fig:gamhs}.

The optional $\gamma$
depended on the closure
and the correction function, $\Delta(r)$.
%
For the almost exact closure, ``BS8'',
with $\Delta(r) = h(r) \, B(r)$,
the low density limit for $\gamma$
is indeed $2/3$,
in agreement with the graphical analysis.
%
This is, however, not so
for the PY and HC equations.
%
The optimal $\gamma$ did appear to be
relatively stable:
$\gamma^\mathrm{id}$ was around
$0.67$ or $0.5$,
for $\Delta = h \, B$
and $\Delta = -B$,
respectively.


\begin{figure}[h]
  \makebox[\linewidth][c]{
    \includegraphics[angle=0, width=0.9\linewidth]{fig/gamhs.pdf}
  }
  \caption{
    \label{fig:gamhs}
    Ideal $\gamma$ values
    for various integral equations
    on the three-dimensional hard-sphere fluid.
    %
    BS8: the closure with the bridge function being
    $B(r) = \sum_{n = 4}^8 B_n(r) \, \rho^{n-2}$.
    %
    The curves for BS8 is limited to $\rho \le 0.7$,
    for higher densities require more terms
    of the series.
  }
\end{figure}




\subsection{Lennard-Jones fluid}




In Fig. \ref{fig:muljpycv},
we show that the chemical potential
computed from the PY equation
on the Lennard-Jones (LJ) fluid
at $T = 1.5$.
%
We studied only the compressibility and virial routes
because of numerical difficulties.
%
It became more difficult
to approximate the virial-route chemical potential
by Eq. \eqref{eq:muex_trial}
than the hard-sphere case.



%
\begin{figure}[h]
  \makebox[\linewidth][c]{
    \includegraphics[angle=0, width=0.9\linewidth]{fig/muljpycv.pdf}
  }
  \caption{
    \label{fig:muljpycv}
    Chemical potential from the PY equation
    on the three-dimensional LJ fluid at $T = 1.5$.
    %
    EOS: the PVE-hBH equation of state\cite{kolafa1994}.
  }
\end{figure}





\section{Conclusions}





We studied
four routes of computing
chemical potential:
the compressibility, virial,
chemical-potential,
and generalized energy routes.
%
The first two routes
tends to be most convenient and stable.



The chemical-potential route
depends on the choice of
the charging parameter, $\xi$,
and each $\xi$ generally
yields a distinct value.
%
The best $\xi$ among the tested choices
is the scaling factor
of the interaction radius
for the hard-sphere fluid.
%
For the HNC equation,
however,
all chemical-potential routes
converge to the virial route.



The generalized energy route
depends on a parameter, $\lambda$,
similar to $\xi$.
%
If $\lambda$ happens to be
the radial scaling factor,
then the generalized energy route
recovers the virial route.
%
Further, for the HNC equation,
all generalized energy routes
converge to the virial route.



The trial functional strategy can usually
give good approximations
for the chemical potential
on the three-dimensional hard-sphere fluid.
%
For more complex fluids, however,
better approximations
to the trial functional
or the correction function, $\Delta$,
might be needed.




\bibliography{liquid}
\end{document}


%\documentclass{revtex4-1}
% options: aip, jcp, reprint, preprint
\documentclass[preprint]{revtex4-1}
%\documentclass[notitlepage, preprint]{revtex4-1}
%\documentclass[aip,jcp,reprint,superscriptaddress]{revtex4-1}

\usepackage{amsmath}
\usepackage{amsthm}
\usepackage{mathrsfs}
\usepackage{graphicx}
\usepackage{grffile}
\usepackage{dcolumn}
\usepackage{bm}
\usepackage{multirow}
\usepackage{hyperref}
\usepackage{tikz}
%\usepackage{setspace}

\tikzstyle{blackdot}=[circle,draw=black,fill=black,
                      inner sep=0pt,minimum size=1.5mm]
\tikzstyle{whitedot}=[circle,draw=black,fill=white,
                      inner sep=0pt,minimum size=1.5mm]
\tikzstyle{combine}=[green!30!black, very thick, densely dotted]
\tikzstyle{concat}=[blue!40!black, very thick, densely dashed]

%\renewcommand{\theequation}{N\arabic{equation}}

\newtheorem{defn}{Definition}
\newtheorem{thrm}{Theorem}
\newtheorem{lemm}[thrm]{Lemma}
\newtheorem{prop}[thrm]{Proposition}

\newcommand{\vct}[1]{\mathbf{#1}}
\providecommand{\vr}{} % clear \vr
\renewcommand{\vr}{\vct{r}}
\newcommand{\vk}{\vct{k}}
\newcommand{\vR}{\vct{R}}
\newcommand{\dvk}{\frac{d\vk}{(2\pi)^D}}
\newcommand{\tdvk}{\tfrac{d\vk}{(2\pi)^D}}

% add a superscript ``ex''
\newcommand{\supex}[1]{ { { #1 }^{ \mathrm{ex} } } }
\newcommand{\Pex}{\supex{P}}
\newcommand{\Ptex}{P^{ \mathrm{ex} }_t}
\newcommand{\Fex}{\supex{F}}
\newcommand{\muex}{\supex{\mu}}
\newcommand{\mutex}{\mu^{ \mathrm{ex} }_t}
\newcommand{\kex}{\supex{\kappa}}
\newcommand{\Chn}{\mathscr{C}}
%\newcommand{\Chn}{\mathcal{C}}
%\newcommand{\Chn}{\mathsf{C}}
\newcommand{\secref}[1]{Sec. \ref{#1}}

\newcommand{\gmin}{g_\mathrm{min}}
\newcommand{\rmin}{r_\mathrm{min}}

\newcommand{\llbra}{[\![}
\newcommand{\llket}{]\!]}





\begin{document}

\title{Computation of chemical potential and pressure from integral equations}

\begin{abstract}
  A simple method is proposed to approximately compute chemical potential and pressure
  from integral equations.
\end{abstract}
%\maketitle



%\section{Introduction}

\section{Methods}

\subsection{Overview}

We wish to determine the ideal parameter $\gamma$
of the following empirical formula of chemical potential $\muex$
%
\begin{align}
  -\beta \muex
  &\approx
    \rho \, \int d\vr \,
    \left[
      c(r) - B(r) - \frac{1}{2} \, t(r) \, h(r)
      -\gamma \, B(r) \, h(r)
    \right].
  \label{eq:mutiapprox}
\end{align}
%The method will be tested for the Lennard-Jones (LJ) fluid.
%
%The parameter $\gamma$ was found to be around $0.6$ in all tested cases.



The ideal $\gamma$ is defined as
%
\begin{equation}
  \gamma =
\frac{
  \beta \muex + \rho \, \int d\vr \,
    \left[
      c(r) - B(r) - \frac{1}{2} \, t(r) \, h(r)
    \right]
  }
  {
    \rho \, \int d\vr \, B(r) \, h(r)
  },
  \label{eq:solvegamma}
\end{equation}
with everything on the right hand side assuming the exact values.

To compute the exact values, we perform
a Monte Carlo (MC) or molecular dynamics (MD) simulation,
and do the following three things.
%
\begin{enumerate}
%
%
%
\item
Compute $-\beta \, \muex$
from Widom's insertion method in the simulation:
\begin{equation}
  \exp( - \beta \, \muex)  = \langle \exp(-\beta \, \Delta U) \rangle,
  \label{eq:widom}
\end{equation}
where $\Delta U$ is the increment of the potential energy
of inserting a particle at a random position.
%
%
%
\item
Deduce $B(r)$ from the radial distribution function $g(r)$
obtained from the simulation
(see details below).
%
%
%
\item
Use $B(r)$ in the closure,
\[
  c(r) = \exp[-\beta \, \phi(r) + t(r) + B(r)] - 1 - t(r),
\]
and use the closure along with the Ornstein-Zernike (OZ) relation
\begin{equation}
  t(r) = \rho (c * h) (r).
  \label{eq:oz}
\end{equation}
to solve $c(r)$, and $t(r)$.
\end{enumerate}

The first and third steps are straightforward.
Below we shall focus on the second step
  of compute the bridge function $B(r)$.




\subsection{Compute the bridge function}

In an MC or MD simulation, we can readily compute
  the radial distribution function $g(r)$.
%
Thus, the most direct method of computing $B(r)$
is to deduce it from $g(r)$:
\begin{equation}
  B(r) = \log y(r) - t(r),
  \label{eq:Brdef}
\end{equation}
where
\[
  y(r) = g(r) \, \exp \beta \phi(r).
\]

The indirect correlation function $t(r)$ is computed as follows.
%
First, from the pair correlation function we get $h(r) = g(r) - 1$,
and its Fourier transform $\tilde h(k)$.
%
Then, we can invert the OZ relation as
\[
  \tilde c(k)
= \frac{ \tilde h(k) }
  { 1 + \rho \, \tilde h(k) }.
\]
The inverse Fourier transform of $\tilde c(k)$ is $c(r)$,
and $t(r) = h(r) - c(r)$.
%
With $t(r)$, we can use Eq. \eqref{eq:Brdef} to find $B(r)$.


Now in the above method,
although $B(r)$ appears in two places in Eq. \eqref{eq:mutiapprox},
it only affects the last term.
%
To see this, we use Eq. \eqref{eq:Brdef} in Eq. \eqref{eq:mutiapprox}, and
%
\begin{align}
  -\beta \muex
  &\approx
    \int d\vr \,
    \left[
      h(r) - \log y(r) - \frac{1}{2} \, t(r) \, h(r)
      -\gamma \, B(r) \, h(r)
    \right].
  \label{eq:mutiapprox2}
\end{align}
%
%Usually, only the short-range behavior of $B(r)$ affects the chemical potential,
%for $h(r) \rightarrow 0$ as $r \rightarrow \infty$.



\subsection{\label{sec:Brextrapolation}Extrapolation}

The above method unfortunately fails for a small $r$,
because $g(r)$ is nearly zero there,
which makes the logarithm in Eq. \eqref{eq:Brdef} very inaccurate.
%
Thus, we used the above method only for $r > \rmin$,
where $\rmin$ is defined as the first positive root of
$g(\rmin) = \gmin$, with $\gmin \approx 0.3$.


A simple remedy is to extrapolate the known $B(r)$ of $r > \rmin$ to $r = 0$.
%
We used the quadratic approximation:
\begin{equation}
  B_\mathrm{fit}(r) \approx a_0 + a_1 r + a_2 r^2,
  \label{eq:Brpoly}
\end{equation}
with the parameters $a_0$, $a_1$ and $a_2$
determined from fitting $B_\mathrm{fit}(r)$ against $B(r)$
for the range $\rmin < r < r_1$,
with $r_1$ set to 1.3.
%
Higher order polynomials tend to cause over-fitting
and thus are unused.
%
The bridge function from Eq. \eqref{eq:Brpoly} is used for $r < \rmin$.

Note that for a continuous molecular potential,
we have $B'(0) = 0$ by the zero separation theorems\cite{meeron1968, zhou1988}.
%
The fitting result from Eq. \eqref{eq:Brpoly} does not satisfy this relation.
%
However, for the purpose of computing the chemical potential,
$B(r)$ appears with the spherical area $d\vr \propto r^2 \, dr$,
which makes the small $r$ behavior unimportant.
%
Thus, Eq. \eqref{eq:Brpoly} is still useful.



\subsection{\label{sec:Brti}Thermodynamic integration}

Another way to compute $B(r)$ is through a separate free energy calculation.
%
We used thermodynamic integration to compute $y(r)$,
which is based on the following identity:
\begin{equation}
  (\log y)'(r_{12})
=
  \frac{1}{2}
  \left.
    \left\langle
      \frac{ \beta \, \vct F_{12} \cdot \vr_{12} }
           { r^2_{12} }
    \right\rangle
  \right|_{r_{12}}.
  \label{eq:dydr}
\end{equation}
where
$\vr_{12} = \vr_1 - \vr_2$,
$\vct F_{12} = \vct F_1 - \vct F_2$ is the difference of the force,
and the average is performed at a fixed distance $r_{12}$.
%
Here, particles 1 and 2 are special and not interacting with each other.
That is, they are replaced by two cavities of unit radius.
%
Then $\log y$ can be computed from integrating $(\log y)'(r_{12})$ from $r_{12} = \infty$.

We can use a special MC simulation to sample a flat histogram along $r_{12}$
to get the profile of $\log y(r)$.
%
The mean force [the right-hand side of Eq. \eqref{eq:dydr}]
is collected in the simulation,
and the profile $\log y(r)$
can be obtained on the fly
by integrating Eq. \eqref{eq:dydr}.



Although the method is more accurate, it requires an extra simulation.
%
The results, however, were very similar to those from the above extrapolation method.





\section{Results}

We applied the method on the LJ fluid.
%
The results are summarized in Table \ref{tab:results}.
%
%We used the standard equation of state from literature\cite{johnson1993} for $\muex$.
%
The radial distribution functions
were collected from MC simulations in the canonical $NVT$ ensemble.
%
In each simulation,
we also computed $\muex$ using Widom's method.
%
Ideally, we should use the grand canonical $\mu VT$ ensemble.
%
However, with a proper scaling (see Appendix \ref{apd:canonscaling}),
the radial distribution function from the $NVT$ ensemble can be corrected to approximate
that in the $\mu VT$ ensemble (see Fig. \ref{fig:rdfscale} for example).
%
For simplicity, we mainly used the $NVT$ ensemble in the production runs.

\begin{figure}[h]
  \makebox[\linewidth][c]{
    \includegraphics[angle=0, width=0.75\linewidth]{../data/lj/rdf/rdf.pdf}
  }
  \caption{
    \label{fig:rdfscale}
    LJ fluid, $T = 10$, $\rho = 0.1$, $N = 256$.
  }
\end{figure}



\subsection{Bridge function}

The $B(r)$ and $g(r)$ in a typical case are shown in Fig. \ref{fig:ljBrT1rho0.7}.
%
The $B(r)$ from our method is close to PY result.
%
There is, however, an artifact:
under a low temperature and high density ($T = 1$, $\rho \ge 0.7$),
$B(r) \, 4 \pi \, r^2$, but not $B(r)$, has a non-negligible oscillating tail.
%
Fortunately, this problem is unlikely to affect the chemical potential too much,
because $B(r)$ enters Eq. \eqref{eq:mutiapprox2} only through
the combination of $B(r) h(r)$,
and $h(r)$ suppresses the oscillation at large $r$.
%
In practice, we found that $B(r) \, h(r) \, 4 \pi \, r^2$ is always short-ranged,
as shown in the inset of panel (a) of Fig. \ref{fig:ljBrT1rho0.7}.
%
Further, when we explicitly set $B(r) = 0$ for $r > r_\mathrm{max}$,
we found little difference in the resulting $\gamma$ values.
%
However, to be cautious,
we should consider the possibility that the $B(r)$, and hence the value $\gamma$,
from our method are less reliable for the low-temperature and high-density cases.


\begin{figure}[h]
  \makebox[\linewidth][c]{
    \includegraphics[angle=0, width=0.75\linewidth]{../data/lj/T1rho0.7/ljBrT1rho0.7.pdf}
  }
  \caption{
    \label{fig:ljBrT1rho0.7}
    LJ fluid, $T = 1$, $\rho = 0.7$.
    (a) $B(r)$.
    Inset: $B(r) \, h(r) \, d\vr$ as a function of $r$.
    (b) $g(r)$ computed from different integral equations.
  }
\end{figure}





\begin{table*}
  \begin{tabular}[c]{p{1cm} p{1.1cm} p{1.3cm} p{3cm} p{1.5cm} p{1.3cm} p{1.3cm} p{1.5cm}
    p{2.0cm} p{1.5cm}}
        $T$
    &   $\rho$
    &   $N$
    &   $\beta \, \mu^{\mathrm{ex} *}$
    &   $\beta \, \mu_0^{\mathrm{ex} \P}$
    &   $\Delta^\S$
    &   $\gamma_\mathrm{T.I.}^\dagger$
    &   $\gamma_\mathrm{fit}^\ddagger$
    &   $\beta \, \muex^{(\mathrm{PY}) \#}$
    &   $\beta \, \muex^{(\mathrm{HNC})}$
        \\
    \hline
     1    &   0.676   &  2048 &   $-3.595$ ($-3.616$)    & $-6.371$  &  $4.47$   &   $0.621$  &   $0.630$  &  $-3.816$ &  $-2.445$  \\
     1    &   0.7     &  2048 &   $-3.458$ ($-3.477$)    & $-6.820$  &  $5.44$   &   $0.618$  &   $0.627$  &  $-3.749$ &  $-2.134$  \\
     $1^\mathrm{GC}$
          &   $0.697$ & $2039.0$& $-3.478$ ($-3.477$)    & $-6.890$  &  $5.57$   &   $0.613$  &   $0.599$  &                        \\
     1    &   0.757   &  2048 &   $-2.916$ ($-2.925$)    & $-7.955$  &  $8.3 $   &   $0.61 $  &   $0.62 $  &  $-3.460$ &  $-1.119$  \\
     1    &   0.8     &  2048 &   $-2.271$ ($-2.272$)    & $-9.121$  &  $11.3$   &   $0.61 $  &   $0.61 $  &  $-3.092$ &  $-0.052$  \\
     $1^\mathrm{GC}$
          &   $0.799$ & $2045.2$& $-2.262$ ($-2.272$)    & $-9.250$  &  $11.5$   &   $0.60 $  &   $0.61 $  &                        \\
     1    &   0.827   &  2048 &   $-1.735$ ($-1.742$)    & $-9.838$  &  $13.5$   &   $0.60 $  &   $0.61 $  &  $-2.783$ &  $+0.768$  \\
    \hline
     1.5  &   0.3     &  864  &   $-1.121$ ($-1.118$)    & $-1.218$  &  $0.129$  &   $0.756$  &   $0.747$  &  $-1.146$ &  $-1.089$  \\
     1.5  &   0.5     &  864  &   $-1.314$ ($-1.322$)    & $-1.936$  &  $0.933$  &   $0.666$  &   $0.664$  &  $-1.387$ &  $-1.052$  \\
     1.5  &   0.7     &  2048 &   $-0.401$ ($-0.400$)    & $-3.148$  &  $4.45$   &   $0.617$  &   $0.623$  &  $-0.732$ &  $+0.638$  \\
     1.5  &   0.8     &  2048 &   $+0.979$ ($+0.955$)    & $-4.255$  &  $8.56$   &   $0.611$  &   $0.614$  &  $+0.240$ &  $+2.650$  \\
    \hline
     2    &   0.3     &  864  &   $-0.477$ ($-0.472$)    & $-0.570$  &  $0.126$  &   $0.74 $  &   $0.72 $  &  $-0.494$ &  $-0.446$  \\
     2    &   0.5     &  864  &   $-0.259$ ($-0.266$)    & $-0.823$  &  $0.860$  &   $0.656$  &   $0.656$  &  $-0.336$ &  $-0.024$  \\
     2    &   0.7     &  2048 &   $+0.990$ ($+1.002$)    & $-1.366$  &  $3.80$   &   $0.620$  &   $0.624$  &  $+0.663$ &  $+1.856$  \\
    \hline
    10    &   0.1     &  864  &   $+0.2136$ ($+0.218$)   & $+0.2117$ &  $0.0024$ &   $0.77 $  &   $0.78 $  &  $+0.2131$&  $+0.2141$ \\
    10    &   0.3     &  864  &   $+0.797$ ($+0.817$)    & $+0.750$  &  $0.070$  &   $0.67 $  &   $0.68 $  &  $+0.786$ &  $+0.817$  \\
    10    &   0.5     &  864  &   $+1.690$ ($+1.717$)    & $+1.424$  &  $0.415$  &   $0.643$  &   $0.646$  &  $+1.633$ &  $+1.795$  \\
    10    &   0.7     &  2048 &   $+3.066$ ($+3.099$)    & $+2.209$  &  $1.359$  &   $0.631$  &   $0.629$  &  $+2.898$ &  $+3.382$  \\
    \hline
    \multicolumn{10}{l}{
      $^*$
      The number was computed from Eq. \eqref{eq:widom},
      and the number in the parentheses is from literature\cite{johnson1993}.
    }
    \\
    \multicolumn{10}{l}{
      $^\P$
      $\beta \mu_0^\mathrm{ex} \equiv \rho \int [B(r) - c(r) + \frac{1}{2} t(r) \, h(r)] \, d\vr$
      with $B(r)$ from the thermodynamic integration method
      (Sec. \ref{sec:Brti}).
    }
    \\
    \multicolumn{10}{l}{
      $^\S$
      $\Delta \equiv \rho \int B(r) \, h(r) \, d\vr$ \
      with $B(r)$ from the thermodynamic integration method
      (Sec. \ref{sec:Brti}).
    }
    \\
    \multicolumn{10}{l}{
      $^\dagger$
      From the thermodynamic integration method
      (Sec. \ref{sec:Brti}).
    }
    \\
    \multicolumn{10}{l}{
      $^\ddagger$
      From the polynomial extrapolation method
      (Sec. \ref{sec:Brextrapolation}).
    }
    \\
    \multicolumn{10}{l}{
      $^\#$
      Using Eq. \eqref{eq:mutiapprox} with $\gamma = 2/3$.
    }
    \\
    \multicolumn{10}{l}{
      $^\mathrm{GC}$
      Grand-canonical ensemble simulations.
    }
    \\
    \hline
  \end{tabular}
  \caption{\label{tab:results}Optimal $\gamma$ for the LJ fluid.}
\end{table*}




\subsection{Ideal $\gamma$}

As shown in Table \ref{tab:results},
the ideal $\gamma$ appears to be around $0.6$ to $0.8$.
%
The value tends to drop with the density, $\rho$.
However, it stays above $0.5$ in all cases.
%
In low density cases,
the correction function $\Delta =\int B(r) \, h(r) \, d\vr$
is very small.
%
Thus, $\gamma$ can be inaccurate in these cases
because $\Delta$ serves as the denominator of \eqref{eq:solvegamma}.
%
It is also clear from the Table that the ensemble difference is trivial.

More data can be found under \texttt{/Bossman/mu/data/lj}.
%
The directories are arranged by the density, $\rho$,
and temperature, $T$,
and each directory contains a figure similar to Fig. \ref{fig:ljBrT1rho0.7}.
%
%The bridge function can be viewed in gnuplot as
%
%\texttt{plot "Brout1.dat" u 1:2 w l}


\subsection{Error of the bridge function}

In practice, we rarely know the exact bridge function $B(r)$.
%
Thus, we wish to know how the ideal $\gamma$ varies
with respect to the error of $B(r)$.
%
For simplicity,
we investigated the case of a scaled bridge function $\hat B(r) = s \, B(r)$,
%
such that the error is represented by $|s - 1|$.
%
For each $s$,
we use $\hat B(r)$ as the bridge function in the integral equation,
and solve for $c(r)$ and $t(r)$.
%
Then, we use the above correlation functions in Eq. \eqref{eq:solvegamma}
to compute the corresponding ideal $\gamma$.

\begin{figure}[h]
  \makebox[\linewidth][c]{
    \includegraphics[angle=0, width=0.75\linewidth]{../data/lj/Bscale/Bscale.pdf}
  }
  \caption{
    \label{fig:Bscale}
    (a) Ideal $\gamma$ as a function of the scaling factor $s$ of the bridge function $B(r)$.
    (b) $\beta \muex$ computed from the scaled bridge function $s \, B(r)$, using $\gamma = 0$ or $2/3$.
  }
\end{figure}


As shown in Fig. \ref{fig:Bscale}(a),
for a sufficiently large scaling $s$,
the ideal $\gamma$ stays around 0.5 to 0.7,
which agrees with our previous observation in Table \ref{tab:results}.
%
However, if the scaling $s$ is too small (i.e., HNC case),
the $\gamma$ can change more dramatically.

If we use a scaled bridge function,
the chemical potential from Eq. \eqref{eq:mutiapprox}
as a function of $s$ is plotted in Fig. \ref{fig:Bscale}(b)
for two empirical values $\gamma = 0$ and $2/3$.
%
It is clear that $\gamma = 2/3$ yields
more accurate and stable (i.e., with a less steep slope) results
than $\gamma = 0$ in most cases.
%
This suggests that
if we use an inexact bridge function,
the empirical value $\gamma = 2/3$
may be preferred over $\gamma = 0$.



\appendix

\section{\label{apd:canonscaling}
Normalization of the radial distribution function $g(r)$}

Because of to the finite size effect,
the radial distribution function $g(r)$
obtained from a simulation in the canonical ensemble
does not approach unity as $r$ approaches the half box size $L/2$,
even in the low-density limit.
Instead, $g(L/2)$ is $1 + O(1/N)$.
%
The $g(r)$ of the grand-canonical ensemble does not suffer from the problem.
%
Below, we present an approximation to normalize the $g(r)$
from a simulation performed in the canonical ensemble,
such that the result would resemble the $g(r)$ in the grand-canonical ensemble.
%
This is based on a technique used in reference\cite{kolafa2002}.


We should normalize $g(r)$ to $g^*(r)$,
such that $h^*(r) \equiv g^*(r) - 1$ satisfies the relation:
\begin{equation}
  1 + \rho \int h^*(r) \, d\vr= \kappa,
  \label{eq:hint}
\end{equation}
where
\begin{align*}
  \kappa
&=
  \frac{1}{N}
  \frac { \partial N }
  { \partial (\beta \, \mu) }
= \frac { \partial \rho }
        { \partial (\beta \, P) }.
\end{align*}
In terms of $g^*(r)$, we have
\[
  \rho \int g^*(r) \, d\vr = N - 1 + \kappa.
\]

In the canonical ensemble, the $g(r)$ is normalized such that
\[
  \rho \int g(r) \, d\vr = N - 1,
\]
Now to make \eqref{eq:hint} true, we scale the $g(r)$
in the canonical ensemble as
\[
  g^*(r) = g(r)
  \left(
  1 +
  \frac{ \kappa }{ N - 1}
  \right),
\]
where
\begin{align*}
\kappa^{-1}
&= 1
+ \left(1-\frac{1}{D}\right) \frac{ \beta \, \Pex }{ \rho}
- \frac{V}{\rho}
  \Bigl\langle
    \Delta (\beta \, \Pex)^2
  \Bigr\rangle
+ \frac{ \beta \, X_2 } { D^2 },
\end{align*}
with $D$ being the dimension, $\Pex$ being the excess chemical potential,
\begin{align*}
  \beta \, \Pex
&=
\rho
-
\frac{ \beta } { D \, V } \sum_{i = 1}^N
\left\langle
  \vr_i \cdot \frac{ \partial U } { \partial \vr_i }
\right\rangle
\\
&=
\rho
-
\frac{ \beta \rho^2 } { 2 D }
\int
r \, \phi'(r) \, d\vr,
\end{align*}
and
\begin{align*}
X_2
&= \frac{1}{N} \, \sum_{i, j = 1}^N
\left\langle
  \vr_j \cdot
  \frac{ \partial^2 U }
  { \partial \vr_j \, \partial \vr_i }
  \cdot \vr_i
\right\rangle
\\
&= \frac{1}{N} \, \sum_{i < j}
 r_{ij}^2 \, \phi''(r_{ij})
= \frac{ \rho }{ 2 }
   \int r^2 \, \phi''(r) \, g(r) \, d\vr.
\end{align*}

\bibliography{liquid}
\end{document}

